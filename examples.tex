\documentclass[nonacm]{acmart}

\usepackage{bm}

\newcommand{\x}{\bm{x}}
\newcommand{\y}{\bm{y}}

\title{Examples of LMIs}

\author{Simone Naldi}
\affiliation{
    \institution{Université de Limoges, CNRS, XLIM}
    \city{Limoges}
    \country{France}
}

\author{Mohab Safey El Din}
\affiliation{
    \institution{Sorbonne Université, CNRS, LIP6}
    \city{Paris}
    \country{France}
}

\author{Adrien Taylor}
\affiliation{
    \institution{Inria, École normale supérieure, PSL Research University}
    \city{Paris}
    \country{France}
}

\author{Weijia Wang}
\affiliation{
    \institution{Sorbonne Université, CNRS, LIP6}
    \city{Paris}
    \country{France}
}

\begin{document}

\maketitle

In this document, we provide explicit examples of LMIs
that are considered in our work.
For their context and motivation,
we refer to the documents provided in the same repository.

\begin{itemize}
    \item \textsf{MKN11} denotes the LMI defined by
          \begin{align*}
              A  & =
              \begin{pmatrix}
                  \varepsilon                         & \frac{1}{2}+\frac{3 \varepsilon}{2} & 0           & 0                          & -x_{24}                    & 0                       \\
                  \frac{1}{2}+\frac{3 \varepsilon}{2} & 1+3 \varepsilon                     & 0           & x_{24}                     & 0                          & 0                       \\
                  0                                   & 0                                   & \varepsilon & 0                          & 0                          & 0                       \\
                  0                                   & x_{24}                              & 0           & 3 \varepsilon              & -\frac{3}{2}+3 \varepsilon & \frac{3 \varepsilon}{2} \\
                  -x_{24}                             & 0                                   & 0           & -\frac{3}{2}+3 \varepsilon & 3 \varepsilon              & \frac{3 \varepsilon}{2} \\
                  0                                   & 0                                   & 0           & \frac{3 \varepsilon}{2}    & \frac{3 \varepsilon}{2}    & 1+\varepsilon           \\
              \end{pmatrix} \\
              \y & =\varepsilon, \quad \x=x_{24}.
          \end{align*}
    \item \textsf{RBN11} denotes the LMI defined by
          \begin{align*}
              A  & =
              \begin{pmatrix}
                  1                      & -\frac{1}{2}+2\epsilon & 0 & 0                               & -x_{24}                         & 0                      \\
                  -\frac{1}{2}+2\epsilon & -1+4\epsilon           & 0 & x_{24}                          & 0                               & 0                      \\
                  0                      & 0                      & 1 & 0                               & 0                               & 0                      \\
                  0                      & x_{24}                 & 0 & -1+4\epsilon                    & \frac{3}{2}+\frac{3}{2}\epsilon & -\frac{1}{2}+2\epsilon \\
                  -x_{24}                & 0                      & 0 & \frac{3}{2}+\frac{3}{2}\epsilon & -1+4\epsilon                    & -\frac{1}{2}+2\epsilon \\
                  0                      & 0                      & 0 & -\frac{1}{2}+2\epsilon          & -\frac{1}{2}+2\epsilon          & 1
              \end{pmatrix} \\
              \y & =\varepsilon, \quad \x=x_{24}.
          \end{align*}
    \item \textsf{GRD24} denotes the LMI defined by
          \begin{equation*}
              A=
              \begin{pmatrix}
                  \frac{\mu  L (\lambda_1+\lambda_3+\lambda_5+\lambda_6)}{L-\mu }                & \star                                                                                                                                                       & \star                                                  \\
                  -\frac{\lambda_5 \mu +L (\gamma  \mu  (\lambda_1+\lambda_6)+\lambda_3)}{L-\mu} & \frac{\lambda_2+\lambda_3+\lambda_4+\lambda_5+\gamma  \mu  (\gamma  L (\lambda_1+\lambda_2+\lambda_4+\lambda_6)-2 \lambda_2)-2 \gamma  \lambda_4 L}{L-\mu } & \star                                                  \\
                  -\frac{\lambda_6 \mu +\lambda_1 L}{L-\mu }                                     & \frac{\gamma  \lambda_4 \mu +\gamma  \lambda_6 \mu -\lambda_2-\lambda_4+\gamma  L (\lambda_1+\lambda_2)}{L-\mu }                                            & \frac{\lambda_1+\lambda_2+\lambda_4+\lambda_6}{L-\mu }
              \end{pmatrix},
          \end{equation*}
          and $\y=(\mu,\tau)$,
          $\x=(\lambda_1,\ldots,\lambda_6)$,
          with the constraints
          $\lambda_1,\ldots,\lambda_6 \geq 0$,
          $\gamma=\frac{2}{L+\mu}$, $L=1$, $0<\mu<1$, and
          \begin{equation*}
              \begin{cases}
                  -\lambda_2+\lambda_3+\lambda_4-\lambda_5+\tau & = 0  \\
                  \lambda_1+\lambda_2-\lambda_4-\lambda_6-1     & = 0,
              \end{cases}
          \end{equation*}
          so that the LMI has 4 free variables,
          say $\lambda_1,\lambda_4,\lambda_5,\lambda_6$,
          and 2 parameters.
    \item \textsf{GRD23} denotes the LMI obtained
          by setting $\lambda_6=0$ in \textsf{GRD24}.
    \item \textsf{GRD14} denotes the LMI obtained
          by moving $\tau$ from $\y$ to $\x$ in \textsf{GRD21}.
    \item \textsf{GRD22} denotes the LMI obtained
          by setting $\lambda_5=0$ in \textsf{GRD23}.
    \item \textsf{GRD13} denotes the LMI obtained
          by moving $\tau$ from $\y$ to $\x$ in \textsf{GRD21}.
    \item \textsf{GRD21} denotes the LMI obtained
          by setting $\lambda_4=0$ in \textsf{GRD22}.
    \item \textsf{GRD12} denotes the LMI obtained
          by moving $\tau$ from $\y$ to $\x$ in \textsf{GRD21}.
    \item
          \textsf{PPM31} denotes the LMI defined by
          \begin{equation*}
              A=
              \begin{pmatrix}
                  2\lambda\mu+2\tau-2            & -\gamma(2\lambda\mu-2)-\lambda         \\
                  -\gamma(2\lambda\mu-2)-\lambda & \gamma(\gamma(2\lambda\mu-2)+2\lambda)
              \end{pmatrix},
          \end{equation*}
          and $\y=(\mu,\gamma,\tau)$, $\x=\lambda$,
          with the constraints $\mu>0$, $\gamma>0$, $\tau>0$.
    \item \textsf{PPM21} denotes the LMI obtained
          by setting $\gamma=1$ in \textsf{PPM31}.
    \item \textsf{DRS42} denotes the LMI proposed in
          \cite[SM3.1.1.]{ryu2020operator},
          defined by
          \begin{equation*}
              A=
              \begin{pmatrix}
                  \rho^2+\beta\lambda_\beta^B-1             & -\theta+\frac{\lambda^A_\mu}{2} & \theta-(\frac{1}{2}+\beta)\lambda_\beta^B \\
                  -\theta+\frac{\lambda^A_\mu}{2}           & -\theta^2+(1+\mu)\lambda^A_\mu  & \theta^2-\lambda^A_\mu                    \\
                  \theta-(\frac{1}{2}+\beta)\lambda_\beta^B & \theta^2-\lambda^A_\mu          & -\theta^2+(1+\beta)\lambda_\beta^B
              \end{pmatrix},
          \end{equation*}
          and $\y=(\mu,\beta,\rho,\theta)$,
          $\x=(\lambda^A_\mu,\lambda_\beta^B)$,
          with the constraints $\mu>0$, $0<\theta<2$, $\beta-\mu\geq 0$.
    \item \textsf{DRS32} denotes the LMI obtained
          by setting $\theta=1$ in \textsf{DRS42}.
    \item \textsf{DRS43} denotes the LMI proposed in
          \cite[SM3.2.2.]{ryu2020operator}, defined by
          \begin{equation*}
              A=
              \begin{pmatrix}
                  \rho^2+\lambda^B_L-1                       & \frac{\lambda^A_\mu}{2}-\theta            & \theta-\lambda^B_L-\frac{\lambda^B_\mu}{2}          \\
                  \frac{\lambda^A_\mu}{2}-\theta             & -\theta^2+\lambda^A_\mu+\lambda^A_\mu \mu & \theta^2-\lambda^A_\mu                              \\
                  \theta-\lambda^B_L-\frac{\lambda^B_\mu}{2} & \theta^2-\lambda^A_\mu                    & -\lambda^B_L L^2-\theta^2+\lambda^B_L+\lambda^B_\mu
              \end{pmatrix},
          \end{equation*}
          and $\y=(\mu,L,\rho,\theta)$,
          $\x=(\lambda^A_\mu,\lambda^B_L,\lambda^B_\mu)$,
          with $\mu>0$, $L-\mu>0$.
    \item \textsf{DRS33} denotes the LMI obtained
          by setting $\theta=1$ in \textsf{DRS43}.
\end{itemize}

\bibliographystyle{plain}
\bibliography{refs}

\end{document}
